\documentclass[12pt]{article} 
\author{Zhenzi Yu, zyu331}
\title{HW1}


\begin{document}
\maketitle

\section*{Question 1}

$$S(p)=\frac{T(n,1)}{T(n,p)}=\frac{\frac{x}{k-m}}{\frac{x}{p*k-m}}=\frac{p*k-m}{k-m}>p$$
\\
\\The reason is that the snow accumulates with the time.

\section*{Question 2}

$$S(p)=\frac{T(n,1)}{T(n,p)}={1/p+1/n}$$
\begin{itemize}
\item fixed n, p continually increase: the speedup continually increases
\item n is proportional to p: the speedup is fixed at 2k/n
\end{itemize}

\section*{Question 3}

\begin{itemize}
\item if $P<n^2$, then both algorithms do not lose efficiency. With $T_A=\frac{n^3}{P}\times log(n$ and $T_B=\frac{n^2}{p}\times n$, so B is always faster
\item if $n^2<P<n^3$
$T_B=n$ and $T_A=\frac{n^3}{p} \times log(n)$, so A is faser
\item if $p>n^3$ $ T_A=log(n)$ and $T_B=n$

\end{itemize}


\section*{Question 4}
(a) $E=\frac{n^2}{p (\frac{n^2}{p}+pn)}=\Theta(1)$ so: $p=O(\sqrt{n})$
\\
(b) to satisfy the memery requirement, $P~O(n)$, if $p=O(\sqrt[n])$ as the effciency requires, the memory tends to be exceeded.

\section*{Question 5}
$$E=\frac{n}{p (\frac{n}{p}+\sqrt{p} \log(p))}=\Theta(1)$$
$$thus: p \sqrt{p} \log{p} =O(n)$$
Guess $$P=O(\frac{n^\frac{2}{3}}{log n^\frac{2}{3}})$$
Subsititute $$\frac{n^\frac{2}{3}}{\log n}^\frac{3}{2} \times (\log n^\frac{2}{3}- \log \log n^\frac{3}{2})=O(1)$$

\end{document}


